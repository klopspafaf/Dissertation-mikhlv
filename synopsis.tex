%&preformat-synopsis
\RequirePackage[l2tabu,orthodox]{nag} % Раскомментировав, можно в логе получать рекомендации относительно правильного использования пакетов и предупреждения об устаревших и нерекомендуемых пакетах

% Откомментируйте, чтобы отключить генерацию закладок в pdf
% \PassOptionsToPackage{bookmarks=false}{hyperref}
\documentclass[a5paper,10pt,twoside,openany,article]{memoir} %,draft

%%%%%%%%%%%%%%%%%%%%%%%%%%%%%%%%%%%%%%%%%%%%%%%%%%%%%%%%%%%%%%%%%%%%%%%%%%%%%%%%
%%%% Файл упрощённых настроек шаблона, общих для диссертации и автореферата %%%%
%%%%%%%%%%%%%%%%%%%%%%%%%%%%%%%%%%%%%%%%%%%%%%%%%%%%%%%%%%%%%%%%%%%%%%%%%%%%%%%%

%%% Режим черновика %%%
\makeatletter
\@ifundefined{c@draft}{
  \newcounter{draft}
  \setcounter{draft}{1}  % 0 --- чистовик (максимальное соблюдение ГОСТ)
                         % 1 --- черновик (отклонения от ГОСТ, но быстрая
                         %       сборка итоговых PDF)
}{}
\makeatother

%%% Пометки в тексте %%%
\makeatletter
\@ifundefined{c@showmarkup}{
  \newcounter{showmarkup}
  \setcounter{showmarkup}{0}  % 0 --- скрыть пометки
                              % 1 --- показывать пометки
}{}
\makeatother

%%% Использование в pdflatex шрифтов не по-умолчанию %%%
\makeatletter
\@ifundefined{c@usealtfont}{
  \newcounter{usealtfont}
  \setcounter{usealtfont}{1}    % 0 --- шрифты на базе Computer Modern
                                % 1 --- использовать пакет pscyr, при его
                                %       наличии
                                % 2 --- использовать пакет XCharter, при наличии
                                %       подходящей версии
}{}
\makeatother

%%% Использование в xelatex и lualatex семейств шрифтов %%%
\makeatletter
\@ifundefined{c@fontfamily}{
  \newcounter{fontfamily}
  \setcounter{fontfamily}{1}  % 0 --- CMU семейство. Используется как fallback;
                              % 1 --- Шрифты от MS (Times New Roman и компания)
                              % 2 --- Семейство Liberation
}{}
\makeatother

%%% Библиография %%%
\makeatletter
\@ifundefined{c@bibliosel}{
  \newcounter{bibliosel}
  \setcounter{bibliosel}{1}   % 0 --- встроенная реализация с загрузкой файла
                              %       через движок bibtex8;
                              % 1 --- реализация пакетом biblatex через движок
                              %       biber
}{}
\makeatother

%%% Вывод типов ссылок в библиографии %%%
\makeatletter
\@ifundefined{c@mediadisplay}{
  \newcounter{mediadisplay}
  \setcounter{mediadisplay}{1}   % 0 --- не делать ничего; надписи [Текст] и
                                 %       [Эл. ресурс] будут выводиться только в ссылках с
                                 %       заполненным полем `media`;
                                 % 1 --- автоматически добавлять надпись [Текст] к ссылкам с
                                 %       незаполненным полем `media`; таким образом, у всех
                                 %       источников будет указан тип, что соответствует
                                 %       требованиям ГОСТ
                                 % 2 --- автоматически удалять надписи [Текст], [Эл. Ресурс] и др.;
                                 %       не соответствует ГОСТ
                                 % 3 --- автоматически удалять надпись [Текст];
                                 %       не соответствует ГОСТ
                                 % 4 --- автоматически удалять надпись [Эл. Ресурс];
                                 %       не соответствует ГОСТ
}{}
\makeatother

%%% Предкомпиляция tikz рисунков для ускорения работы %%%
\makeatletter
\@ifundefined{c@imgprecompile}{
  \newcounter{imgprecompile}
  \setcounter{imgprecompile}{0}   % 0 --- без предкомпиляции;
                                  % 1 --- пользоваться предварительно
                                  %       скомпилированными pdf вместо генерации
                                  %       заново из tikz
}{}
\makeatother
          % общие настройки шаблона
\input{common/packages}       % Пакеты общие для диссертации и автореферата
\synopsistrue                 % Этот документ --- автореферат
\input{Synopsis/synpackages}  % Пакеты для автореферата
\usepackage{fr-longtable}    %ради \endlasthead

% Листинги с исходным кодом программ
\usepackage{fancyvrb}
\usepackage{listings}
\lccode`\~=0\relax %Без этого хака из-за особенностей пакета listings перестают работать конструкции с \MakeLowercase и т. п. в (xe|lua)latex

% Русская традиция начертания греческих букв
\usepackage{upgreek} % прямые греческие ради русской традиции

%Мои пакеты
\usepackage{braket}

%%% Микротипографика
%\ifnumequal{\value{draft}}{0}{% Только если у нас режим чистовика
%    \usepackage[final, babel, shrink=45]{microtype}[2016/05/14] % улучшает представление букв и слов в строках, может помочь при наличии отдельно висящих слов
%}{}

% Отметка о версии черновика на каждой странице
% Чтобы работало надо в своей локальной копии по инструкции
% https://www.ctan.org/pkg/gitinfo2 создать небходимые файлы в папке
% ./git/hooks
% If you’re familiar with tweaking git, you can probably work it out for
% yourself. If not, I suggest you follow these steps:
% 1. First, you need a git repository and working tree. For this example,
% let’s suppose that the root of the working tree is in ~/compsci
% 2. Copy the file post-xxx-sample.txt (which is in the same folder of
% your TEX distribution as this pdf) into the git hooks directory in your
% working copy. In our example case, you should end up with a file called
% ~/compsci/.git/hooks/post-checkout
% 3. If you’re using a unix-like system, don’t forget to make the file executable.
% Just how you do this is outside the scope of this manual, but one
% possible way is with commands such as this:
% chmod g+x post-checkout.
% 4. Test your setup with “git checkout master” (or another suitable branch
% name). This should generate copies of gitHeadInfo.gin in the directories
% you intended.
% 5. Now make two more copies of this file in the same directory (hooks),
% calling them post-commit and post-merge, and you’re done. As before,
% users of unix-like systems should ensure these files are marked as
% executable.
\ifnumequal{\value{draft}}{1}{% Черновик
   \IfFileExists{.git/gitHeadInfo.gin}{
      \usepackage[mark,pcount]{gitinfo2}
      \renewcommand{\gitMark}{rev.\gitAbbrevHash\quad\gitCommitterEmail\quad\gitAuthorIsoDate}
      \renewcommand{\gitMarkFormat}{\rmfamily\color{Gray}\small\bfseries}
   }{}
}{} % Пакеты для специфических пользовательских задач

% Новые переменные, которые могут использоваться во всём проекте
% ГОСТ 7.0.11-2011
% 9.2 Оформление текста автореферата диссертации
% 9.2.1 Общая характеристика работы включает в себя следующие основные структурные
% элементы:
% актуальность темы исследования;
\newcommand{\actualityTXT}{Актуальность темы.}
% степень ее разработанности;
\newcommand{\progressTXT}{Степень разработанности темы.}
% цели и задачи;
\newcommand{\aimTXT}{Целью}
\newcommand{\tasksTXT}{задачи}
% научную новизну;
\newcommand{\noveltyTXT}{Научная новизна:}
% теоретическую и практическую значимость работы;
%\newcommand{\influenceTXT}{Теоретическая и практическая значимость}
% или чаще используют просто
\newcommand{\influenceTXT}{Практическая значимость}
% методологию и методы исследования;
\newcommand{\methodsTXT}{Методология и методы исследования.}
% положения, выносимые на защиту;
\newcommand{\defpositionsTXT}{Основные положения, выносимые на~защиту:}
% степень достоверности и апробацию результатов.
\newcommand{\reliabilityTXT}{Достоверность}
\newcommand{\probationTXT}{Апробация работы.}

\newcommand{\contributionTXT}{Личный вклад.}
\newcommand{\publicationsTXT}{Публикации.}

\newcommand{\SixJ}[6]{\begin{Bmatrix} #1 & #2 & #3
\\#4 & #5 & #6\end{Bmatrix}}
\newcommand{\TreJ}[6]{\begin{pmatrix} #1 & #2 & #3
\\#4 & #5 & #6\end{pmatrix}}                                    
\newcommand{\xiB}{\boldsymbol{\xi}}
\newcommand{\Tprod}[2]{\left\{#1\otimes#2\right\}}
\newcommand{\rhosm}{\rho^{(-)}}
\newcommand{\rhosp}{\rho^{(+)}}
\newcommand{\rhosz}{\rho^{(0)}}
\newcommand{\rhosmDO}{\rho^{(-21)}}
\newcommand{\rhosmOD}{\rho^{(-12)}} 
\newcommand{\rhospDO}{\rho^{(+21)}}
\newcommand{\rhospOD}{\rho^{(+12)}}
\newcommand\bra[2][]{#1\langle {#2} #1\rvert}
\newcommand\ket[2][]{#1\lvert {#2} #1\rangle}
\newcommand{\Wp}[1]{w(p_{#1}(\delta))}
\newcommand{\Wpc}[1]{w(p_{#1}(\delta))^*}
\newcommand{\p}[1]{p_{#1}(\delta)}
\newcommand{\pc}[1]{p_{#1}(\delta)^*}
%%% Заголовки библиографии:

% для автореферата:
\newcommand{\bibtitleauthor}{Публикации автора по теме диссертации}

% для стиля библиографии `\insertbiblioauthorgrouped`
\newcommand{\bibtitleauthorvak}{В изданиях из списка ВАК РФ}
\newcommand{\bibtitleauthorscopus}{В изданиях, входящих в международную базу цитирования Scopus}
\newcommand{\bibtitleauthorwos}{В изданиях, входящих в международную базу цитирования Web of Science}
\newcommand{\bibtitleauthorother}{В прочих изданиях}
\newcommand{\bibtitleauthorconf}{В сборниках трудов конференций}
\newcommand{\bibtitleauthorpatent}{Зарегистрированные патенты}
\newcommand{\bibtitleauthorprogram}{Зарегистрированные программы для ЭВМ}

% для стиля библиографии `\insertbiblioauthorimportant`:
\newcommand{\bibtitleauthorimportant}{Наиболее значимые \protect\MakeLowercase\bibtitleauthor}

% для списка литературы в диссертации и списка чужих работ в автореферате:
\newcommand{\bibtitlefull}{Список литературы} % (ГОСТ Р 7.0.11-2011, 4)
       % Новые переменные, которые могут использоваться во всём проекте
%%%%%%%%%%%%%%%%%%%%%%%%%%%%%%%%%%%%%%%%%%%%%%%%%%%%%%%%%%%%%%%%%%%%%%%%%%%%%%%%
%%%% Файл упрощённых настроек шаблона, общих для диссертации и автореферата %%%%
%%%%%%%%%%%%%%%%%%%%%%%%%%%%%%%%%%%%%%%%%%%%%%%%%%%%%%%%%%%%%%%%%%%%%%%%%%%%%%%%

%%% Режим черновика %%%
\makeatletter
\@ifundefined{c@draft}{
  \newcounter{draft}
  \setcounter{draft}{1}  % 0 --- чистовик (максимальное соблюдение ГОСТ)
                         % 1 --- черновик (отклонения от ГОСТ, но быстрая
                         %       сборка итоговых PDF)
}{}
\makeatother

%%% Пометки в тексте %%%
\makeatletter
\@ifundefined{c@showmarkup}{
  \newcounter{showmarkup}
  \setcounter{showmarkup}{0}  % 0 --- скрыть пометки
                              % 1 --- показывать пометки
}{}
\makeatother

%%% Использование в pdflatex шрифтов не по-умолчанию %%%
\makeatletter
\@ifundefined{c@usealtfont}{
  \newcounter{usealtfont}
  \setcounter{usealtfont}{1}    % 0 --- шрифты на базе Computer Modern
                                % 1 --- использовать пакет pscyr, при его
                                %       наличии
                                % 2 --- использовать пакет XCharter, при наличии
                                %       подходящей версии
}{}
\makeatother

%%% Использование в xelatex и lualatex семейств шрифтов %%%
\makeatletter
\@ifundefined{c@fontfamily}{
  \newcounter{fontfamily}
  \setcounter{fontfamily}{1}  % 0 --- CMU семейство. Используется как fallback;
                              % 1 --- Шрифты от MS (Times New Roman и компания)
                              % 2 --- Семейство Liberation
}{}
\makeatother

%%% Библиография %%%
\makeatletter
\@ifundefined{c@bibliosel}{
  \newcounter{bibliosel}
  \setcounter{bibliosel}{1}   % 0 --- встроенная реализация с загрузкой файла
                              %       через движок bibtex8;
                              % 1 --- реализация пакетом biblatex через движок
                              %       biber
}{}
\makeatother

%%% Вывод типов ссылок в библиографии %%%
\makeatletter
\@ifundefined{c@mediadisplay}{
  \newcounter{mediadisplay}
  \setcounter{mediadisplay}{1}   % 0 --- не делать ничего; надписи [Текст] и
                                 %       [Эл. ресурс] будут выводиться только в ссылках с
                                 %       заполненным полем `media`;
                                 % 1 --- автоматически добавлять надпись [Текст] к ссылкам с
                                 %       незаполненным полем `media`; таким образом, у всех
                                 %       источников будет указан тип, что соответствует
                                 %       требованиям ГОСТ
                                 % 2 --- автоматически удалять надписи [Текст], [Эл. Ресурс] и др.;
                                 %       не соответствует ГОСТ
                                 % 3 --- автоматически удалять надпись [Текст];
                                 %       не соответствует ГОСТ
                                 % 4 --- автоматически удалять надпись [Эл. Ресурс];
                                 %       не соответствует ГОСТ
}{}
\makeatother

%%% Предкомпиляция tikz рисунков для ускорения работы %%%
\makeatletter
\@ifundefined{c@imgprecompile}{
  \newcounter{imgprecompile}
  \setcounter{imgprecompile}{0}   % 0 --- без предкомпиляции;
                                  % 1 --- пользоваться предварительно
                                  %       скомпилированными pdf вместо генерации
                                  %       заново из tikz
}{}
\makeatother
        % Упрощённые настройки шаблона

%%% Основные сведения %%%
\newcommand{\thesisAuthorLastName}{Михайлов}
\newcommand{\thesisAuthorOtherNames}{Антон Михайлович}
\newcommand{\thesisAuthorInitials}{А.\,М.}
\newcommand{\thesisAuthor}             % Диссертация, ФИО автора
{%
    \texorpdfstring{% \texorpdfstring takes two arguments and uses the first for (La)TeX and the second for pdf
        \thesisAuthorLastName~\thesisAuthorOtherNames% так будет отображаться на титульном листе или в тексте, где будет использоваться переменная
    }{%
        \thesisAuthorLastName, \thesisAuthorOtherNames% эта запись для свойств pdf-файла. В таком виде, если pdf будет обработан программами для сбора библиографических сведений, будет правильно представлена фамилия.
    }
}
\newcommand{\thesisAuthorShort}        % Диссертация, ФИО автора инициалами
{\thesisAuthorInitials~\thesisAuthorLastName}
%\newcommand{\thesisUdk}                % Диссертация, УДК
%{\fixme{xxx.xxx}}
\newcommand{\thesisTitle}              % Диссертация, название
{Теория субдоплеровских резонансов в поле встречных поляризованных бихроматических лазерных пучков}
\newcommand{\thesisSpecialtyNumber}    % Диссертация, специальность, номер
{1.3.19}
\newcommand{\thesisSpecialtyTitle}     % Диссертация, специальность, название (название взято с сайта ВАК для примера)
{Лазерная физика}
%% \newcommand{\thesisSpecialtyTwoNumber} % Диссертация, вторая специальность, номер
%% {\fixme{XX.XX.XX}}
%% \newcommand{\thesisSpecialtyTwoTitle}  % Диссертация, вторая специальность, название
%% {\fixme{Теория и~методика физического воспитания, спортивной тренировки,
%% оздоровительной и~адаптивной физической культуры}}
\newcommand{\thesisDegree}             % Диссертация, ученая степень
{кандидата физико-математических наук}
\newcommand{\thesisDegreeShort}        % Диссертация, ученая степень, краткая запись
{канд. физ.-мат. наук}
\newcommand{\thesisCity}               % Диссертация, город написания диссертации
{Новосибирск}
\newcommand{\thesisYear}               % Диссертация, год написания диссертации
{\the\year}
\newcommand{\thesisOrganization}       % Диссертация, организация
{Федеральное государственное бюджетное образовательное учреждение высшего профессионального образования <<Новосибирский национальный исследовательский государственный университет <<НГУ>>}
\newcommand{\thesisOrganizationShort}  % Диссертация, краткое название организации для доклада
{\fixme{НазУчДисРаб}}

\newcommand{\thesisInOrganization}     % Диссертация, организация в предложном падеже: Работа выполнена в ...
{институте лазерной физики СО РАН}

%% \newcommand{\supervisorDead}{}           % Рисовать рамку вокруг фамилии
\newcommand{\supervisorFio}              % Научный руководитель, ФИО
{Бражников Денис Викторович}
\newcommand{\supervisorRegalia}          % Научный руководитель, регалии
{к.ф.-м.н., \fixme{уч. звание}}
\newcommand{\supervisorFioShort}         % Научный руководитель, ФИО
{Д.\,В.~Бражников}
\newcommand{\supervisorRegaliaShort}     % Научный руководитель, регалии
{\fixme{к.ф.-м.н.,~уч.~зв.}}

%% \newcommand{\supervisorTwoDead}{}        % Рисовать рамку вокруг фамилии
%% \newcommand{\supervisorTwoFio}           % Второй научный руководитель, ФИО
%% {\fixme{Фамилия Имя Отчество}}
%% \newcommand{\supervisorTwoRegalia}       % Второй научный руководитель, регалии
%% {\fixme{уч. степень, уч. звание}}
%% \newcommand{\supervisorTwoFioShort}      % Второй научный руководитель, ФИО
%% {\fixme{И.\,О.~Фамилия}}
%% \newcommand{\supervisorTwoRegaliaShort}  % Второй научный руководитель, регалии
%% {\fixme{уч.~ст.,~уч.~зв.}}

\newcommand{\opponentOneFio}           % Оппонент 1, ФИО
{\fixme{Фамилия Имя Отчество}}
\newcommand{\opponentOneRegalia}       % Оппонент 1, регалии
{\fixme{доктор физико-математических наук, профессор}}
\newcommand{\opponentOneJobPlace}      % Оппонент 1, место работы
{\fixme{Не очень длинное название для места работы}}
\newcommand{\opponentOneJobPost}       % Оппонент 1, должность
{\fixme{старший научный сотрудник}}

\newcommand{\opponentTwoFio}           % Оппонент 2, ФИО
{\fixme{Фамилия Имя Отчество}}
\newcommand{\opponentTwoRegalia}       % Оппонент 2, регалии
{\fixme{кандидат физико-математических наук}}
\newcommand{\opponentTwoJobPlace}      % Оппонент 2, место работы
{\fixme{Основное место работы c длинным длинным длинным длинным названием}}
\newcommand{\opponentTwoJobPost}       % Оппонент 2, должность
{\fixme{старший научный сотрудник}}

%% \newcommand{\opponentThreeFio}         % Оппонент 3, ФИО
%% {\fixme{Фамилия Имя Отчество}}
%% \newcommand{\opponentThreeRegalia}     % Оппонент 3, регалии
%% {\fixme{кандидат физико-математических наук}}
%% \newcommand{\opponentThreeJobPlace}    % Оппонент 3, место работы
%% {\fixme{Основное место работы c длинным длинным длинным длинным названием}}
%% \newcommand{\opponentThreeJobPost}     % Оппонент 3, должность
%% {\fixme{старший научный сотрудник}}

\newcommand{\leadingOrganizationTitle} % Ведущая организация, дополнительные строки. Удалить, чтобы не отображать в автореферате
{\fixme{Федеральное государственное бюджетное образовательное учреждение высшего
профессионального образования с~длинным длинным длинным длинным названием}}

\newcommand{\defenseDate}              % Защита, дата
{\fixme{DD mmmmmmmm YYYY~г.~в~XX часов}}
\newcommand{\defenseCouncilNumber}     % Защита, номер диссертационного совета
{\fixme{Д\,123.456.78}}
\newcommand{\defenseCouncilTitle}      % Защита, учреждение диссертационного совета
{\fixme{Название учреждения}}
\newcommand{\defenseCouncilAddress}    % Защита, адрес учреждение диссертационного совета
{\fixme{Адрес}}
\newcommand{\defenseCouncilPhone}      % Телефон для справок
{\fixme{+7~(0000)~00-00-00}}

\newcommand{\defenseSecretaryFio}      % Секретарь диссертационного совета, ФИО
{\fixme{Фамилия Имя Отчество}}
\newcommand{\defenseSecretaryRegalia}  % Секретарь диссертационного совета, регалии
{\fixme{д-р~физ.-мат. наук}}            % Для сокращений есть ГОСТы, например: ГОСТ Р 7.0.12-2011 + http://base.garant.ru/179724/#block_30000

\newcommand{\synopsisLibrary}          % Автореферат, название библиотеки
{\fixme{Название библиотеки}}
\newcommand{\synopsisDate}             % Автореферат, дата рассылки
{\fixme{DD mmmmmmmm}\the\year~года}

% To avoid conflict with beamer class use \providecommand
\providecommand{\keywords}%            % Ключевые слова для метаданных PDF диссертации и автореферата
{}
           % Основные сведения
\input{common/fonts}          % Определение шрифтов (частичное)
\input{common/styles}         % Стили общие для диссертации и автореферата
\input{Synopsis/synstyles}    % Стили для автореферата
\input{Synopsis/userstyles}   % Стили для специфических пользовательских задач

%%% Библиография. Выбор движка для реализации %%%
\ifnumequal{\value{bibliosel}}{0}{%
    \input{biblio/predefined} % Встроенная реализация с загрузкой файла через движок bibtex8
}{
    \input{biblio/biblatex}   % Реализация пакетом biblatex через движок biber
}

% Вывести информацию о выбранных опциях в лог сборки
\typeout{Selected options:}
\typeout{Draft mode: \arabic{draft}}
\typeout{Font: \arabic{fontfamily}}
\typeout{AltFont: \arabic{usealtfont}}
\typeout{Bibliography backend: \arabic{bibliosel}}
\typeout{Precompile images:  \arabic{imgprecompile}}
% Вывести информацию о версиях используемых библиотек в лог сборки
\listfiles

\begin{document}

\input{Synopsis/title}        % Титульный лист
%\mainmatter                   % В том числе начинает нумерацию страниц арабскими цифрами с единицы
\mainmatter*                  % Нумерация страниц не изменится, но начнётся с новой страницы
\pdfbookmark{Общая характеристика работы}{characteristic}             % Закладка pdf
\section*{Общая характеристика работы}

\newcommand{\actuality}{\pdfbookmark[1]{Актуальность}{actuality}\underline{\textbf{\actualityTXT}}}
\newcommand{\progress}{\pdfbookmark[1]{Разработанность темы}{progress}\underline{\textbf{\progressTXT}}}
\newcommand{\aim}{\pdfbookmark[1]{Цели}{aim}\underline{{\textbf\aimTXT}}}
\newcommand{\tasks}{\pdfbookmark[1]{Задачи}{tasks}\underline{\textbf{\tasksTXT}}}
\newcommand{\aimtasks}{\pdfbookmark[1]{Цели и задачи}{aimtasks}\aimtasksTXT}
\newcommand{\novelty}{\pdfbookmark[1]{Научная новизна}{novelty}\underline{\textbf{\noveltyTXT}}}
\newcommand{\influence}{\pdfbookmark[1]{Практическая значимость}{influence}\underline{\textbf{\influenceTXT}}}
\newcommand{\methods}{\pdfbookmark[1]{Методология и методы исследования}{methods}\underline{\textbf{\methodsTXT}}}
\newcommand{\defpositions}{\pdfbookmark[1]{Положения, выносимые на защиту}{defpositions}\underline{\textbf{\defpositionsTXT}}}
\newcommand{\reliability}{\pdfbookmark[1]{Достоверность}{reliability}\underline{\textbf{\reliabilityTXT}}}
\newcommand{\probation}{\pdfbookmark[1]{Апробация}{probation}\underline{\textbf{\probationTXT}}}
\newcommand{\contribution}{\pdfbookmark[1]{Личный вклад}{contribution}\underline{\textbf{\contributionTXT}}}
\newcommand{\publications}{\pdfbookmark[1]{Публикации}{publications}\underline{\textbf{\publicationsTXT}}}


{\actuality} Обзор, введение в тему, обозначение места данной работы в
мировых исследованиях и~т.\:п., можно использовать ссылки на~другие
работы~\autocite{Gosele1999161,Lermontov}
(если их~нет, то~в~автореферате
автоматически пропадёт раздел <<Список литературы>>). Внимание! Ссылки
на~другие работы в~разделе общей характеристики работы можно
использовать только при использовании \verb!biblatex! (из-за технических
ограничений \verb!bibtex8!. Это связано с тем, что одна
и~та~же~характеристика используются и~в~тексте диссертации, и в
автореферате. В~последнем, согласно ГОСТ, должен присутствовать список
работ автора по~теме диссертации, а~\verb!bibtex8! не~умеет выводить в~одном
файле два списка литературы).
При использовании \verb!biblatex! возможно использование исключительно
в~автореферате подстрочных ссылок
для других работ командой \verb!\autocite!, а~также цитирование
собственных работ командой \verb!\cite!. Для этого в~файле
\verb!common/setup.tex! необходимо присвоить положительное значение
счётчику \verb!\setcounter{usefootcite}{1}!.

Для генерации содержимого титульного листа автореферата, диссертации апрошвшпорвкп
и~презентации используются данные из файла \verb!common/data.tex!. Если,
например, вы меняете название диссертации, то оно автоматически
появится в~итоговых файлах после очередного запуска \LaTeX. Согласно
ГОСТ 7.0.11-2011 <<5.1.1 Титульный лист является первой страницей
диссертации, служит источником информации, необходимой для обработки и
поиска документа>>. Наличие логотипа организации на~титульном листе а
упрощает обработку и~поиск, для этого разметите логотип вашей хххххххххххххххххх лололололллолололо
организации в папке images в~формате PDF (лучше найти его в векторном
варианте, чтобы он хорошо смотрелся при печати) под именем
\verb!logo.pdf!. Настроить размер изображения с логотипом можно
в~соответствующих местах файлов \verb!title.tex!  отдельно для tt
диссертации и автореферата. Если вам логотип не~нужен, то просто
удалите файл с~логотипом.

\ifsynopsis
Этот абзац появляется только в~автореферате.
Для формирования блоков, которые будут обрабатываться только в~автореферате,
заведена проверка условия \verb!\!\verb!ifsynopsis!.
Значение условия задаётся в~основном файле документа (\verb!synopsis.tex! для
автореферата).
\else
Этот абзац появляется только в~диссертации.
Через проверку условия \verb!\!\verb!ifsynopsis!, задаваемого в~основном файле
документа (\verb!dissertation.tex! для диссертации), можно сделать новую
команду, обеспечивающую появление цитаты в~диссертации, но~не~в~автореферате.
\fi

% {\progress}
% Этот раздел должен быть отдельным структурным элементом по
% ГОСТ, но он, как правило, включается в описание актуальности
% темы. Нужен он отдельным структурынм элемементом или нет ---
% смотрите другие диссертации вашего совета, скорее всего не нужен.

{\aim} данной работы является \ldots

Для~достижения поставленной цели необходимо было решить следующие {\tasks}:
\begin{enumerate}[beginpenalty=10000] % https://tex.stackexchange.com/a/476052/104425
  \item Исследовать, разработать, вычислить и~т.\:д. и~т.\:п.
  \item Исследовать, разработать, вычислить и~т.\:д. и~т.\:п.
  \item Исследовать, разработать, вычислить и~т.\:д. и~т.\:п.
  \item Исследовать, разработать, вычислить и~т.\:д. и~т.\:п.
\end{enumerate}


{\novelty}
\begin{enumerate}[beginpenalty=10000] % https://tex.stackexchange.com/a/476052/104425
  \item Впервые \ldots
  \item Впервые \ldots
  \item Было выполнено оригинальное исследование \ldots
\end{enumerate}

{\influence} \ldots

{\methods} \ldots

{\defpositions}
\begin{enumerate}[beginpenalty=10000] % https://tex.stackexchange.com/a/476052/104425
  \item Первое положение
  \item Второе положение
  \item Третье положение
  \item Четвертое положение
\end{enumerate}
В папке Documents можно ознакомиться с решением совета из Томского~ГУ
(в~файле \verb+Def_positions.pdf+), где обоснованно даются рекомендации
по~формулировкам защищаемых положений.

{\reliability} полученных результатов обеспечивается \ldots \ Результаты находятся в соответствии с результатами, полученными другими авторами.


{\probation}
Основные результаты работы докладывались~на:
перечисление основных конференций, симпозиумов и~т.\:п.

{\contribution} Автор принимал активное участие \ldots

\ifnumequal{\value{bibliosel}}{0}
{%%% Встроенная реализация с загрузкой файла через движок bibtex8. (При желании, внутри можно использовать обычные ссылки, наподобие `\cite{vakbib1,vakbib2}`).
    {\publications} Основные результаты по теме диссертации изложены
    в~XX~печатных изданиях,
    X из которых изданы в журналах, рекомендованных ВАК,
    X "--- в тезисах докладов.
}%
{%%% Реализация пакетом biblatex через движок biber
    \begin{refsection}[bl-author, bl-registered]
        % Это refsection=1.
        % Процитированные здесь работы:
        %  * подсчитываются, для автоматического составления фразы "Основные результаты ..."
        %  * попадают в авторскую библиографию, при usefootcite==0 и стиле `\insertbiblioauthor` или `\insertbiblioauthorgrouped`
        %  * нумеруются там в зависимости от порядка команд `\printbibliography` в этом разделе.
        %  * при использовании `\insertbiblioauthorgrouped`, порядок команд `\printbibliography` в нём должен быть тем же (см. biblio/biblatex.tex)
        %
        % Невидимый библиографический список для подсчёта количества публикаций:
        \printbibliography[heading=nobibheading, section=1, env=countauthorvak,          keyword=biblioauthorvak]%
        \printbibliography[heading=nobibheading, section=1, env=countauthorwos,          keyword=biblioauthorwos]%
        \printbibliography[heading=nobibheading, section=1, env=countauthorscopus,       keyword=biblioauthorscopus]%
        \printbibliography[heading=nobibheading, section=1, env=countauthorconf,         keyword=biblioauthorconf]%
        \printbibliography[heading=nobibheading, section=1, env=countauthorother,        keyword=biblioauthorother]%
        \printbibliography[heading=nobibheading, section=1, env=countregistered,         keyword=biblioregistered]%
        \printbibliography[heading=nobibheading, section=1, env=countauthorpatent,       keyword=biblioauthorpatent]%
        \printbibliography[heading=nobibheading, section=1, env=countauthorprogram,      keyword=biblioauthorprogram]%
        \printbibliography[heading=nobibheading, section=1, env=countauthor,             keyword=biblioauthor]%
        \printbibliography[heading=nobibheading, section=1, env=countauthorvakscopuswos, filter=vakscopuswos]%
        \printbibliography[heading=nobibheading, section=1, env=countauthorscopuswos,    filter=scopuswos]%
        %
        \nocite{*}%
        %
        {\publications} Основные результаты по теме диссертации изложены в~\arabic{citeauthor}~печатных изданиях,
        \arabic{citeauthorvak} из которых изданы в журналах, рекомендованных ВАК\sloppy%
        \ifnum \value{citeauthorscopuswos}>0%
            , \arabic{citeauthorscopuswos} "--- в~периодических научных журналах, индексируемых Web of~Science и Scopus\sloppy%
        \fi%
        \ifnum \value{citeauthorconf}>0%
            , \arabic{citeauthorconf} "--- в~тезисах докладов.
        \else%
            .
        \fi%
        \ifnum \value{citeregistered}=1%
            \ifnum \value{citeauthorpatent}=1%
                Зарегистрирован \arabic{citeauthorpatent} патент.
            \fi%
            \ifnum \value{citeauthorprogram}=1%
                Зарегистрирована \arabic{citeauthorprogram} программа для ЭВМ.
            \fi%
        \fi%
        \ifnum \value{citeregistered}>1%
            Зарегистрированы\ %
            \ifnum \value{citeauthorpatent}>0%
            \formbytotal{citeauthorpatent}{патент}{}{а}{}\sloppy%
            \ifnum \value{citeauthorprogram}=0 . \else \ и~\fi%
            \fi%
            \ifnum \value{citeauthorprogram}>0%
            \formbytotal{citeauthorprogram}{программ}{а}{ы}{} для ЭВМ.
            \fi%
        \fi%
        % К публикациям, в которых излагаются основные научные результаты диссертации на соискание учёной
        % степени, в рецензируемых изданиях приравниваются патенты на изобретения, патенты (свидетельства) на
        % полезную модель, патенты на промышленный образец, патенты на селекционные достижения, свидетельства
        % на программу для электронных вычислительных машин, базу данных, топологию интегральных микросхем,
        % зарегистрированные в установленном порядке.(в ред. Постановления Правительства РФ от 21.04.2016 N 335)
    \end{refsection}%
    \begin{refsection}[bl-author, bl-registered]
        % Это refsection=2.
        % Процитированные здесь работы:
        %  * попадают в авторскую библиографию, при usefootcite==0 и стиле `\insertbiblioauthorimportant`.
        %  * ни на что не влияют в противном случае
        \nocite{vakbib2}%vak
        \nocite{patbib1}%patent
        \nocite{progbib1}%program
        \nocite{bib1}%other
        \nocite{confbib1}%conf
    \end{refsection}%
        %
        % Всё, что вне этих двух refsection, это refsection=0,
        %  * для диссертации - это нормальные ссылки, попадающие в обычную библиографию
        %  * для автореферата:
        %     * при usefootcite==0, ссылка корректно сработает только для источника из `external.bib`. Для своих работ --- напечатает "[0]" (и даже Warning не вылезет).
        %     * при usefootcite==1, ссылка сработает нормально. В авторской библиографии будут только процитированные в refsection=0 работы.
}

При использовании пакета \verb!biblatex! будут подсчитаны все работы, добавленные
в файл \verb!biblio/author.bib!. Для правильного подсчёта работ в~различных
системах цитирования требуется использовать поля:
\begin{itemize}
        \item \texttt{authorvak} если публикация индексирована ВАК,
        \item \texttt{authorscopus} если публикация индексирована Scopus,
        \item \texttt{authorwos} если публикация индексирована Web of Science,
        \item \texttt{authorconf} для докладов конференций,
        \item \texttt{authorpatent} для патентов,
        \item \texttt{authorprogram} для зарегистрированных программ для ЭВМ,
        \item \texttt{authorother} для других публикаций.
\end{itemize}
Для подсчёта используются счётчики:
\begin{itemize}
        \item \texttt{citeauthorvak} для работ, индексируемых ВАК,
        \item \texttt{citeauthorscopus} для работ, индексируемых Scopus,
        \item \texttt{citeauthorwos} для работ, индексируемых Web of Science,
        \item \texttt{citeauthorvakscopuswos} для работ, индексируемых одной из трёх баз,
        \item \texttt{citeauthorscopuswos} для работ, индексируемых Scopus или Web of~Science,
        \item \texttt{citeauthorconf} для докладов на конференциях,
        \item \texttt{citeauthorother} для остальных работ,
        \item \texttt{citeauthorpatent} для патентов,
        \item \texttt{citeauthorprogram} для зарегистрированных программ для ЭВМ,
        \item \texttt{citeauthor} для суммарного количества работ.
\end{itemize}
% Счётчик \texttt{citeexternal} используется для подсчёта процитированных публикаций;
% \texttt{citeregistered} "--- для подсчёта суммарного количества патентов и программ для ЭВМ.

Для добавления в список публикаций автора работ, которые не были процитированы в
автореферате, требуется их~перечислить с использованием команды \verb!\nocite! в
\verb!Synopsis/content.tex!.
 % Характеристика работы по структуре во введении и в автореферате не отличается (ГОСТ Р 7.0.11, пункты 5.3.1 и 9.2.1), потому её загружаем из одного и того же внешнего файла, предварительно задав форму выделения некоторым параметрам

%Диссертационная работа была выполнена при поддержке грантов \dots

%\underline{\textbf{Объем и структура работы.}} Диссертация состоит из~введения,
%четырех глав, заключения и~приложения. Полный объем диссертации
%\textbf{ХХХ}~страниц текста с~\textbf{ХХ}~рисунками и~5~таблицами. Список
%литературы содержит \textbf{ХХX}~наименование.

\pdfbookmark{Содержание работы}{description}                          % Закладка pdf
\section*{Содержание работы}
Во \underline{\textbf{введении}} обосновывается актуальность
исследований, проводимых в~рамках данной диссертационной работы,
приводится обзор научной литературы по~изучаемой проблеме,
формулируется цель, ставятся задачи работы, излагается научная новизна
и практическая значимость представляемой работы. В~последующих главах
сначала описывается общий принцип, позволяющий \dots, а~потом идёт
апробация на частных примерах: \dots  и~\dots.


\underline{\textbf{Первая глава}} посвящена \dots

картинку можно добавить так:
\begin{figure}[ht]
    \centerfloat{
        \hfill
        \subcaptionbox{\LaTeX}{%
            \includegraphics[scale=0.27]{latex}}
        \hfill
        \subcaptionbox{Knuth}{%
            \includegraphics[width=0.25\linewidth]{knuth1}}
        \hfill
    }
    \caption{Подпись к картинке.}\label{fig:latex}
\end{figure}

Формулы в строку без номера добавляются так:
\[
    \lambda_{T_s} = K_x\frac{d{x}}{d{T_s}}, \qquad
    \lambda_{q_s} = K_x\frac{d{x}}{d{q_s}},
\]

\underline{\textbf{Вторая глава}} посвящена исследованию

\underline{\textbf{Третья глава}} посвящена исследованию

Можно сослаться на свои работы в автореферате. Для этого в файле
\verb!Synopsis/setup.tex! необходимо присвоить положительное значение
счётчику \verb!\setcounter{usefootcite}{1}!. В таком случае ссылки на
работы других авторов будут подстрочными.
Изложенные в третьей главе результаты опубликованы в~\cite{}.
Использование подстрочных ссылок внутри таблиц может вызывать проблемы.

В \underline{\textbf{четвертой главе}} приведено описание

\FloatBarrier
\pdfbookmark{Заключение}{conclusion}                                  % Закладка pdf
В \underline{\textbf{заключении}} приведены основные результаты работы, которые заключаются в следующем:
\input{common/concl}

\pdfbookmark{Литература}{bibliography}                                % Закладка pdf
При использовании пакета \verb!biblatex! список публикаций автора по теме
диссертации формируется в разделе <<\publications>>\ файла
\verb!common/characteristic.tex!  при помощи команды \verb!\nocite!

\ifdefmacro{\microtypesetup}{\microtypesetup{protrusion=false}}{} % не рекомендуется применять пакет микротипографики к автоматически генерируемому списку литературы
\urlstyle{rm}                               % ссылки URL обычным шрифтом
\ifnumequal{\value{bibliosel}}{0}{% Встроенная реализация с загрузкой файла через движок bibtex8
    \renewcommand{\bibname}{\large \bibtitleauthor}
    \nocite{*}
    \insertbiblioauthor           % Подключаем Bib-базы
    %\insertbiblioexternal   % !!! bibtex не умеет работать с несколькими библиографиями !!!
}{% Реализация пакетом biblatex через движок biber
    % Цитирования.
    %  * Порядок перечисления определяет порядок в библиографии (только внутри подраздела, если `\insertbiblioauthorgrouped`).
    %  * Если не соблюдать порядок "как для \printbibliography", нумерация в `\insertbiblioauthor` будет кривой.
    %  * Если цитировать каждый источник отдельной командой --- найти некоторые ошибки будет проще.
    %
    %% authorvak
    \nocite{}%
    \nocite{}%
    %
    %% authorwos
    \nocite{}%
    %
    %% authorscopus
    \nocite{}%
    %
    %% authorpathent
    \nocite{}%
    %
    %% authorprogram
    \nocite{}%
    %
    %% authorconf
    \nocite{}%
    \nocite{}%
    %
    %% authorother
    \nocite{}%
    \nocite{}%

    \ifnumgreater{\value{usefootcite}}{0}{
        \begin{refcontext}[labelprefix={}]
            \ifnum \value{bibgrouped}>0
                \insertbiblioauthorgrouped    % Вывод всех работ автора, сгруппированных по источникам
            \else
                \insertbiblioauthor      % Вывод всех работ автора
            \fi
        \end{refcontext}
    }{
        \ifnum \totvalue{citeexternal}>0
            \begin{refcontext}[labelprefix=A]
                \ifnum \value{bibgrouped}>0
                    \insertbiblioauthorgrouped    % Вывод всех работ автора, сгруппированных по источникам
                \else
                    \insertbiblioauthor      % Вывод всех работ автора
                \fi
            \end{refcontext}
        \else
            \ifnum \value{bibgrouped}>0
                \insertbiblioauthorgrouped    % Вывод всех работ автора, сгруппированных по источникам
            \else
                \insertbiblioauthor      % Вывод всех работ автора
            \fi
        \fi
        %  \insertbiblioauthorimportant  % Вывод наиболее значимых работ автора (определяется в файле characteristic во второй section)
        \begin{refcontext}[labelprefix={}]
            \insertbiblioexternal            % Вывод списка литературы, на которую ссылались в тексте автореферата
        \end{refcontext}
        % Невидимый библиографический список для подсчёта количества внешних публикаций
        % Используется, чтобы убрать приставку "А" у работ автора, если в автореферате нет
        % цитирований внешних источников.
        \printbibliography[heading=nobibheading, section=0, env=countexternal, keyword=biblioexternal, resetnumbers=true]%
    }
}
\ifdefmacro{\microtypesetup}{\microtypesetup{protrusion=true}}{}
\urlstyle{tt}                               % возвращаем установки шрифта ссылок URL
      % Содержание автореферата

%%% Выходные сведения типографии
\newpage\thispagestyle{empty}

\vspace*{0pt plus1fill}

\small
\begin{center}
    \textit{\thesisAuthor}
    \par\medskip

    \thesisTitle
    \par\medskip

    Автореф. дис. на соискание ученой степени \thesisDegreeShort
    \par\bigskip

    Подписано в печать \blank[\widthof{999}].\blank[\widthof{999}].\blank[\widthof{99999}].
    Заказ № \blank[\widthof{999999999999}]

    Формат 60\(\times\)90/16. Усл. печ. л. 1. Тираж 100 экз.
    %Это не совсем формат А5, но наиболее близкий, подробнее: http://ru.wikipedia.org/w/index.php?oldid=78976454

    Типография \blank[0.5\linewidth]
\end{center}
\cleardoublepage

\end{document}
